

 % !TEX encoding = UTF-8 Unicode

\documentclass[a4paper]{report}

\usepackage[T2A]{fontenc} % enable Cyrillic fonts
\usepackage[utf8x,utf8]{inputenc} % make weird characters work
\usepackage[serbian]{babel}
%\usepackage[english,serbianc]{babel}
\usepackage{amssymb}

\usepackage{color}
\usepackage{url}
\usepackage[unicode]{hyperref}
\hypersetup{colorlinks,citecolor=green,filecolor=green,linkcolor=blue,urlcolor=blue}

\newcommand{\odgovor}[1]{\textcolor{blue}{#1}}

\begin{document}

\title{Doktorske studije informatike u Srbiji}

%\\ \small{Dopunite autore rada}}

\author{Ivan Pop-Jovanov, Tatjana Kunić, Viktor Novaković, Pavle Cvejović\\ mi18085@alas.matf.bg.ac.rs, mi17139@alas.matf.bg.ac.rs, \\mi18092@alas.matf.bg.ac.rs, mi18024@alas.matf.bg.ac.rs}

\maketitle

\tableofcontents
 
\chapter{Recenzent \odgovor{--- ocena: 5} }


\section{O čemu rad govori?}
% Напишете један кратак пасус у којим ћете својим речима препричати суштину рада (и тиме показати да сте рад пажљиво прочитали и разумели). Обим од 200 до 400 карактера.
U okviru rada prikazane su doktorske studije informatike u Srbiji. Akcenat je stavlje na uslove za upis, cene studija i razne pravilnike. Manje je diskutovano o suštini doktorskih studija, šta se sa njima postiže i čemu one u stvarnosti služe.

\section{Krupne primedbe i sugestije}

Bilo bi dobro da se delovi prepisani iz pravilnika skrate, a da se diskusije na suštinske teme pojačaju. 

\odgovor{U kom smislu pojačaju? Da budu duži? Nismo baš sigurni na šta se ovde misli pod "{}suštinskim temama"{}.}

Veoma mi se sviđa što su autori u okviru rada sami uradili istraživanja i napravili grafik koristeći informacije koje su prilikom svojih istraživanja sakupili. 

% Напишете своја запажања и конструктивне идеје шта у раду недостаје и шта би требало да се промени-измени-дода-одузме да би рад био квалитетнији.
\begin{quote}
Doktorske studije informatike su jedinstveni istraživački stepen defini-
san inovativnom primenom i pronalaskom računarskih metoda za unapredenje
postojećih ili novostvorenih oblasti istraživanja. Istraživanje i obrazova-
nje u informatici imaju snažan interdisciplinarni fokus koji spaja osnove
informacija i računarstva sa oblastima primene. 
\end{quote}

\odgovor{Definicija je sada blago izmenjena, hvala na primedbi.}

Meni prethodna definicija uopšte nije jasna i nakon nje ja tek ne znam šta su doktorske studije. Čini mi se da su doktorske studije nastavak osnovnih i master studija, ali koji ima za primarni cilj da se studenti obuče za samostalni istraživački rad. Takođe, naslov koji u sebi sadrži samo jedan pasus ne zaslužuje da bude posebna sekcija na isti način kako što jedna rečenica ne čini pasus. 


\section{Sitne primedbe}
% Напишете своја запажања на тему штампарских-стилских-језичких грешки
\begin{itemize}
\item Sažetak: doktorske akadEmske studije
\odgovor{Prepravljeno.}
\item sreću sa pitanjem „ Šta dalje?” Da li nastaviti svoje
usavršavanje na fakultetu, ili prvom mogućom prilikom ući u industriju? \\
Nije jasno zašto je prvo pitanje pod navodnicima a ostala nisu? To bi trebalo nekako ujednačiti. 
\odgovor{Stavljeni su sada navodnici oko svih pitanja.}
\item individuama --- osobama, individuama mnogo čudno zvuči
\odgovor{Prepravljeno.}
\item informacije na temu --- informacije na TU temu
\odgovor{Prepravljeno.}
\item U ovom poglavlju ćemo ispričati šta su to doktroske studije, kako su
organizovane i koja je njihova struktura kao i koliko je njihovo vreme
trajanja propisano planom i programom fakulteta i koliko vremena zaista
treba studentima da bi ih zavšili.\\
Deluje kao da je to već rečeno i da je u pitanju ponavljanje. Jer, već svaki sledeći podnaslov sadrži iste ove inforamcije tako da nije potrebno da se one ponavljaju. 
\odgovor{Odlučili smo da ipak ostane ovako kako je, jer daje potreban kontekst na mestima gde je potreban, i podosta olakšava čitanje, pogotovo ako neko ne čita redom.}
\item od 10 fakulteta u kojima postoji smer --- reći ovde tačan broj, pritom napisati broj slovima
\odgovor{U različitim delovima rada smo koristili različite fakultete, pa možda i ne bi bilo iskreno da stavimo tačan broj.}
\item infromatiku, --- informatiku,
\odgovor{Prepravljeno.}
\item Relevantne informacije smo uzeli sa više
od 10 fakulteta u kojima postoji smer (ili odsek) za infromatiku, neki od
kojih su Elektrotehnički fakultet, Fakultet organizacionih nauka, Mašinski
fakultet u Beogradu, Fakultet za informacione tehnologije i inžinjerstvo,
Prirodno-matematički fakultet u Novom Sadu i Kragujevcu[1, 2, 3, 4]...\\
Rečenica je predugačka, podeliti je na dve. 
Takođe, staviti citat uy svaki fakultet da bi se videlo koja referenca se odnosi na šta.
Ako se kaže ''neko od njih su'' onda nema potrebe stavljati tri tačke na kraju. 
\odgovor{Prepravljeno.}
\item Šta su to doktorske studije?\\
Naslov ne bi trebalo da bude pitanje --- to ne odgovara akademskom stilu pisanja, više odgovara novinskom stilu. 
\odgovor{Izbačen je podnaslov i pasus je pridružen prethodnoj sekciji.}
\item Doktorske studije su na većini gorepomenutih fakulteta \\
Gorepomenutih se ne koristi u akademskom pisanju jer nije jasno šta je dole, šta je gore, ako je drugačiji prelom stvari mogu i da se izmeste i da ne budu gore ili dole. Zato je potrebno da se ovo referiše direktno ili nekako drugačije da se formuliše rečenica. 
\odgovor{Zamenjena reč "gorepomenutih"{} sa "razmatranih".}
\item Što se tiče trajanja doktorskih studija po planu i programu fakulteta,
na svim fakultetima pomenutim na početku ovog poglavlja je situacija
ista, doktorske studije traju 3 godine i nose 180 ESPB bodova kao i na
našem, Matematičkom fakultetu u Beogradu[2]. \\
Predugačka rečenica, podeliti je na dve. Takođe, staviti blanko ispred citata.
\odgovor{Prepravljeno.}
\item Da bi lice moglo \\
Ovo ''lice'' jako čudno zvuči.
\odgovor{Stavljeno "osoba"{} umesto "lice"{}.}
\item Univerzitet u Beogradu[19], u članu 6, \\
Dodati blanko ispred citata
\odgovor{Prepravljeno.}
\item Generalno, kod svakog citata je potrebno dodati blanko ispred, to je pogrešno urađeno skoro svuda --- treba svuda ispraviti
\odgovor{Prepravljeno svuda.}
\item Na doktorske studije može konkurisati lice koje ima završene osnovne
akademske i master akademske studije, ili integrisane studije sa najma-
nje 300 ESPB bodova. Takode, može konkurisati i lice koje ima završene
najmanje četvorogodišnje studije po propisima koji su važili do stupanja
na snagu Zakona o visokom obrazovanju („Sl. glasnik RS”, br. 76/2005,
100/2007 - autentično tumačenje, 97/2008, 44/2010, 93/2012, 89/2013,
99/2014, 45/2015 - autentično tumačenje, 68/2015 i 87/2016). \\
Ako je nešto direktno preuzeto treba da bude u quote okruženju. Međutim, ovde stvarno nije potrebno navoditi službeni glasnik i sve ove brojeve, jer je to suvišno i opterećuje čitanje rada. To treba izbaciti. 
\odgovor{Prebačeno u fusnotu.}
\item naulni rezultati. --- ?
\odgovor{Prepravljeno.}
\item 3.2 --- ponovo imamo poglavlje sa samo jednim pasusom. To treba nekako eliminisati.
\odgovor{Poglavlje je blago izmenjeno i podeljeno u dva pasusa.}
\item Na Singidunum univerzitetu, --- Na univerzitetu Singidunum,
\odgovor{Prepravljeno.}
\item 3.5 --- ponovo imamo poglavlje sa samo jednim pasusom. To treba nekako eliminisati.
\odgovor{Sekcija o visinama školarina i sekcija o kvotama za upis su sada spojene u jednu sekciju.}
\item Bilo bi lepo da se informacije iz 3.5 prikažu na nekoj slici. \odgovor{Dodat je chart koji prikazuje cene školarina uređene po visini.}
\item 3.6 --- ponovo imamo poglavlje sa samo jednim pasusom. To treba nekako eliminisati.
\odgovor{Sekcija o visinama školarina i sekcija o kvotama za upis su sada spojene u jednu sekciju.}
\item Bilo bi lepo da se informacije iz 3.6 prikažu u tabeli. 
\odgovor{Tako smo prvobitno i hteli da uradimo, ali kao što ste verovatno i primetili prilikom čitanja rada, nismo uspeli da nađemo informacije na par mesta, ili smo uspeli da nađemo samo za poslednji upisni rok i slično. Zato smo se odlučili da je ipak čitljivije da ne bude u tabeli nego ovako kako je. Ovde ću staviti tabelu koju smo napravili čisto da bi se videlo da nije baš pregledno (Tabela \ref{tab:tabela-budzet-samofinans}). Zvezdicama su obeležena mesta gde smo našli samo informacije za poslednji upisni rok, a upitnicima tamo gde nismo našli informacije.}

\begin{table}[h!]
\begin{center}
\caption{Broj budžetskih i samofinansirajućih mesta}
\begin{tabular}{|l|r|r|} \hline
Fakultet& Budžet& Samofinansiranje\\ \hline
Matematički fakultet& 10& 5\\ \hline
Elektrotehnički fakultet&30&50\\ \hline
Fakultet organizacionih nauka& 0*& 4*\\ \hline
Računarski fakultet& 0& 5 + 5 + 5\\ \hline
Metropolitan& 0& 15\\ \hline
Singidunum& ?& ?\\ \hline
PMF (Novi Sad)& ?& ?\\ \hline
PMF (Kragujevac)& ?& ?\\ \hline
PMF (Niš)& 7*& 3*\\ \hline
\end{tabular}
\label{tab:tabela-budzet-samofinans}
\end{center}
\end{table}

\item Na stranici Katedre za računarstvo i informatiku Matematičkog fakulteta su dobijene sledeće informacije[21].\\
Rečenica ne treba da čini pasus. Takođe, nije jasno šta je tačno od narednih inforamcija odatle preuzeto. 
U nastavku ima još dosta rečenica - pasusa, to treba ispraviti. \odgovor{Prepravljeno.}
\item Za kraj ovog poglavlja, ostaje samo da odgovorimo na pitanje „Koje je najbolje mesto u Srbiji za doktorske studije informatike?”\\
Postoji još dosta pitanja na koja se ovde može odgovoriti, a ovo je samo jedno od njih. U skladu sa time, preformulisati rečenicu. \odgovor{Preformulisano, nova rečenica glasi: Za kraj ovog poglavlja, jedno od pitanja na koje ostaje da odgovorimo je "Koje je najbolje mesto u Srbiji za doktorske studije informatike?"\ }
\item Nije jasno na koji način su podaci iz tabele 1 relevantni za rad? \odgovor{Sada je u rad dodat komentar "{}Iz ovih informacija se vidi da je Univerzitet u Beogradu najpopularniji univerzitet u Srbiji."{}}
\item više kao
nešto što bi eventualno moglo\\
Preformulisati
\odgovor{Prepravljeno. Rečenica je razdvojena u dve.}
\item zaposlenih u IT kompanijama.[18] Jedna od \\
Staviti citat na odgvarajuće mesto.
\odgovor{Prepravljeno.}
\item Ovde sada ide gomila citata na pogrešnim mestima, kao gore, posle kraja rečenice. Srediti sve citate i ujednačiti stil, vidi se da su ovo pisale različite osobe. \odgovor{Prepravljeno.}
\item a pola u industriji na polju istraživanja i razvoja.[10]\\
Po zakonu osoba može da bude zaposlena najviše sa 130 procenata radnog veremena tako da nešto nije u redu sa ovom tvrdnjom.
\odgovor{Nejasno je šta je problem u ovoj tvrdnji. Ako je osoba zaposlena pola radnog vremena na fakultetu, a pola u industriji onda je to 100\% radnog vremena, što je manje od maksimalnih 130\%. Takođe, ova tvrdnja se spominje i u citiranom tekstu. Preformulisali smo sada rečenicu da bude malo jasnija.}
\end{itemize}

\section{Provera sadržajnosti i forme seminarskog rada}
% Oдговорите на следећа питања --- уз сваки одговор дати и образложење



\begin{enumerate}
\item Da li rad dobro odgovara na zadatu temu?\\
Veoma dobro.
\item Da li je nešto važno propušteno?\\
Definicija doktorskih studija nije jasna.
\item Da li ima suštinskih grešaka i propusta?\\
Nema.
\item Da li je naslov rada dobro izabran?\\
Naslov je ok, ali podnaslovi u radu nisu dobro izabrani, kao što je to komentarisano. 
\odgovor{Promenjeni su neki od naslova, izbačeno je pitanje iz naslova.}
\item Da li sažetak sadrži prave podatke o radu?\\
Sažetak bi bilo dobro malo proširiti. 
\odgovor{Proširen je sažetak. Dodate su dve rečenice: "{}Razmatramo različite fakultete u Srbiji i predstavljamo njihovo poređenje po različitim aspektima, kao što su cene školarina doktorskih studija i kvote za broj kandidata za upis. Posebna pažnja je posvećena doktorskim studijama informatike na Matematičkom fakultetu u Beogradu."{}}
\item Da li je rad lak-težak za čitanje?\\
Generalno je ok, ali oni delovi koji su preuzeti iz nekih pravilnika su naporni. 
\odgovor{ Delovi iz pravilnika su u ovom radu podosta sažeti u odnosu na njihov original. Sobzirom da je čitanje originalnih konkursa veoma dug i naporan proces, hteo sam (Ivan) da ovde na jednom mestu prikažem celokupan broj uslova koje bi kandidat mogao da ispunjava pri upisu, da bi neko ko je zainteresovan mogao na brzinu da se informiše o svojim opcijama bez da se udubljuje u originale. Na primer, deo u kome pominjem da je upis omogućen magistrima nauka ima samo dve rečenice, dok u originalu ima četiri paragrafa. Dobar deo napornosti čitanja dolazi iz činjenice da sam u cilju navodjena što preciznijih informacija zadržao dobar deo originalnog formalno-legalnog žargona, ali stojim pri tome da je ovo od velike koristi nekome ko zapravo želi da se prijavi na doktorske studije, a oni koji nisu zainteresovani uvek mogu da samo prelistaju to poglavlje. }
\item Da li je za razumevanje teksta potrebno predznanje i u kolikoj meri?\\
Nije potrebno predznanje.
\item Da li je u radu navedena odgovarajuća literatura?\\
Jeste. Međutim, nesotaje jedna knjiga i jedan naučni časopis. 
\odgovor{Sada je dodato.}
\item Da li su u radu referenceSCI lista je skraćenica za "Current Contents - Science Citation Index" i predstavlja bazu podataka o naučnim radovima objavljenim u različitim časopisima. Koristi se za praćenje citata u naučnim radovima i rangiranje časopisa prema broju citata. Pomaže naučnicima da prate korišćenje njihovih radova i da prate najnovija istraživanja u svojoj oblasti. korektno navedene?\\
Na nekim mestima su reference van rečenica i na nekoliko mesta nedostaje blanko ispred citata.
\odgovor{Prepravljeno.}
\item Da li je struktura rada adekvatna?\\
Postoje naslovi ispod kojih je samo jedan pasus a da pritom to nisu uvodni pasusi već se to odnosi na celokupna podpoglavlja i to bi trebalo izmeniti. Ima i nekoliko rečenica-pasusa. \odgovor{Trebalo bi da su sve rečenice-pasusi i pasusi-sekcije uklonjeni sada. } Trebalo bi malo ukrupniti strukturu. Treće poglavlje ima čudnu strukturu kada se pogleda sadržaj --- ide opšta priča, pa specifična za MatF pa onda opet neka opšta priča. \odgovor{Nije baš tako, počinje pričom o Univerzitetu u Beogradu, onda ide MATF, i na kraju pričamo ukratko o ostalim fakultetima.}
\item Da li rad sadrži sve elemente propisane uslovom seminarskog rada (slike, tabele, broj strana...)?\\
Da, osim literature. \odgovor{Dodato sada.}
\item Da li su slike i tabele funkcionalne i adekvatne?\\
Jesu. Mada nije jasno kako jedna tabela doprinosi smislu rada. \odgovor{Sada je u rad dodat komentar "{}Iz ovih informacija se vidi da je Univerzitet u Beogradu najpopularniji univerzitet u Srbiji."{}}
\end{enumerate}

\section{Ocenite sebe}
% Napišite koliko ste upućeni u oblast koju recenzirate: 
% a) ekspert u datoj oblasti
 b) veoma upućeni u oblast
% c) srednje upućeni
% d) malo upućeni 
% e) skoro neupućeni
% f) potpuno neupućeni
% Obrazložite svoju odluku


\chapter{Recenzent \odgovor{--- ocena: 5} }


\section{O čemu rad govori?}
% Напишете један кратак пасус у којим ћете својим речима препричати суштину рада (и тиме показати да сте рад пажљиво прочитали и разумели). Обим од 200 до 400 карактера.
Suština ovog rada je predstavljanje svih mogućnosti za studente zainteresovane za upis na doktorske 
studije informatike. U radu su spomenuti i delimično opisani svi (zapravo većina, moguće je da je par 
manje 
poznatih studijskih programa izostavljeno) studijski programi doktorskih studija informatike dostupni u 
Srbiji. U radu su takodje navedeni i objašnjeni uslovi upisa na doktorske studije, broj dostupnih mesta, 
visina školarine, sam koncept doktorskih studija informatike. Pri kraju rada je otvoreno pitanje značaja 
diplome doktorskih studija informatike za rad u IT industriji i pružen odgovor na to pitanje. Takođe, 
spomenuta su naučna udruženja i grupe kojima se budući studenti doktorskih studija mogu pridružiti kako 
bi produbili svoja znanja iz oblasti u kojima bi želeli da se dalje usavršavaju i stekli eventualnu neka 
praktična iskustva. Poslednje, opisana je kolaboracija između naučnih, visokoobrazovnih institucija i 
industrijskih korporacija.

\section{Krupne primedbe i sugestije}
% Напишете своја запажања и конструктивне идеје шта у раду недостаје и шта би требало да се промени-измени-дода-одузме да би рад био квалитетнији.
Kupnih primedbi i sugestija nemam, ali da ova sekcija ne ostane prazna ostaviću sledeće...
Mislim da radu nedostaje lični pečat svakog od autora. Znam da bi seminarski radovi trebalo da 
budu informativni i objektivni, ali bilo kakva interpretacija iznesenih podataka mislim da ne bi 
pokvarila kvalitet samog rada. Verujem da bi mišljenje autora bilo skroz osnovano, 
bez obzira da li bi se čitaoci i recenzenti slagali sa njim ili ne, i zato bih voleo da mogu da ga 
pročitam. Ovo bi trebalo da budu samo podstrek da se autori na ovako nešto odluče. Nadam se da 
ovo neće biti na neki drugi način shvaćeno. \odgovor{Nismo dodali ništa novo po pitanju "ličnog pečata autora", jer mislimo da je u redu kako je trenutno.}

\section{Sitne primedbe}
% Напишете своја запажања на тему штампарских-стилских-језичких грешки
Promeniti konstrukciju 'studenti se često sreću sa pitanjem' u uvodu u, na primer, konstrukciju tipa
'studenti u tom periodu sebi postavljaju pitanje'.
\odgovor{Prepravljeno.}


Izmeniti paragraf koji se odnosi na podnaslov 'Uopšteno o doktorskim studijama'. Mislim da bi ideja 
doktorskih studija mogla malo bolje da se opiše. Više instrukcija za ovaj paragraf potražiti pod 
pitanjem 'Da li je rad lak-težak za čitanje?'.

Objasniti šta je SCI lista i eventualno koji žurnali se na njoj nalaze-

\odgovor{Stavljeno u fusnotu: "{}SCI lista je skraćenica za 'Current Contents - Science Citation Index' i predstavlja bazu podataka o naučnim radovima objavljenim u različitim časopisima. Koristi se za praćenje citata u naučnim radovima i rangiranje časopisa prema broju citata."{} }

Doraditi sekciju 'Najbolja mesta za doktorske studije informatike u Srbiji'. Mislim da bi trebalo malo prokomentarisati podatak iz Times Higher Education rang liste. Ovako je samo naveden.

\odgovor{Sada je u rad dodat komentar "{}Iz ovih informacija se vidi da je Univerzitet u Beogradu najpopularniji univerzitet u Srbiji."{}}

Doraditi sekciju 'Znacaj doktorata za rad u industriji' tako da se neodređene konstrukcije tipa 'skoro 
svi oglasi', 'većina studenata' i druge zamene nekim drugim konstrukcijama. Korišćenjem neodređenih 
konstrukcija, stiče se utisak da se neke stvari generalizuju. 

\odgovor{Dodati su podaci koji dokazuju tvrdnje.}

\section{Provera sadržajnosti i forme seminarskog rada}
% Oдговорите на следећа питања --- уз сваки одговор дати и образложење

\begin{enumerate}
\item Da li rad dobro odgovara na zadatu temu?\\

Rad dobro odgovara na temu. Mislim da je tema vrlo bitna, iako možda nezahvalna za autore samog rada u smislu da je 
teže doći do podataka od interesa, i da je ovaj rad neko pre ili kasnije morao napisati, a utisak je 
da to niko do sada nije uradio ili da do takvog rada nije moguće doći. Stoga, mislim da je teško imati neka preterano velika očekivanja od ovog rada (u smislu čega sve rad treba da se dotakne). Međutim, autori su vrlo dobro organizovali rad i, kao što već rekoh, u potpunosti odgovorili na temu.

\item Da li je nešto važno propušteno?\\

Ne bih rekao da je išta krucijalno propušteno u ovom radu. Kako mislim da je pokušaja izrade bilo kakvog 
rada ili teksta na ovu temu bilo malo ili do ovog momomenta možda nije ni bilo, trebalo bi autorima dati 
slobodu da stave akcenat na stvari koje oni smatraju bitnim. Mislim da su autori to vrlo dobro uradili. 
Jedino što bih predložio, a što bi daleko povećalo obim ovog seminarskog rada, jeste sekcija "Motivacija 
za upis na doktorske studije" u kojoj bismo mogli da čujemo utiske nekih trenutnih ili bivših studenata 
doktorskih studija informatike u Srbiji. 

\odgovor{Iako se slažemo da bi bilo lepo imati takvu sekciju, ne mislimo da je u ovom trenutku vremenski izvodljivo.}

\item Da li ima suštinskih grešaka i propusta?\\

Nema ih.

\item Da li je naslov rada dobro izabran?\\

Verujem da bi sadržaj ovog rada zanimao uglavnom pretendente za upis na doktorske studije, te stoga 
mislim da se nikakav senzacionalistički naslov ne bi svideo toj ciljnoj grupi. Takodje, čini se da 
autori rada nisu ni imali mnogo izbora, jer teško da se bilo koja reč sadržana u naslovu može izostaviti 
iz istog.

\item Da li sažetak sadrži prave podatke o radu?\\ 

Rekao bih da ih sadrži. Jedino na šta bih skrenuo pažnju što se uvodnog dela rada tiče, jeste to da su 
sažetak i uvod delimično slični i da bi po trenutnoj verziji rada bilo možda bolje da to što stoji u 
uvodu (osim prvog paragrafa koji bi trebalo da u uvodu i ostane) bude zapravo sažetak, a da sažetak 
posluži kao nastavak za uvod, ali možda malo prepravljen. Uvod bi, po mom mišljenju, trebalo da nas 
priprema za priču koja sledi, a paragrafi u kojima piše šta se u kojoj sekciji rada nalazi bi možda 
trebalo da se nađe u sažetku rada. 

\odgovor{Ne slažemo se, mislimo da paragraf u kom piše šta se nalazi u kojoj sekciji treba da bude u uvodu.}

\item Da li je rad lak-težak za čitanje?\\

Rad je u svom velikom delu lak za čitanje, ali postoje delovi u kojima to baš i nije. Vrlo verovatno su 
te delove pisale različite kolege, ali, svakako, da je ispod naslova rada naveden samo jedan autor niko 
od čitalaca, pa ni sam recenzent ne bi to pomislio. Naglasio bih da pomenuti delovi nisu nečitljivi niti 
teško čitljivi, već je u tim istim delovima iskorišćeno previše priloga i prideva koji su te rečenice 
učinili nejasnim. Deo koji bi možda trebalo izmeniti jeste paragraf ispod naslova "Šta su to doktorske 
studije?".

\odgovor{Paragraf "Šta su to doktorske 
studije?"{} je sada izmenjen.}

\item Da li je za razumevanje teksta potrebno predznanje i u kolikoj meri?\\

Rekao bih da za razumevanje sadržaja ovog seminarskog rada nije potrebno ikakvo predznanje. Rad je 
prijemčiv i za starije i mlađe čitaoce i svako od čitaoca mu može kritički pristupiti. Još jedna pohvala za autore.

\item Da li je u radu navedena odgovarajuća literatura?\\

Rad sadrži odgovarajuću i to pozamašnu literaturu i za to bih pohvalio autore.

\item Da li su u radu reference korektno navedene?\\

U radu su reference ispravno navedene za svaki od korišćenih sajtova i za to bih još jednom pohvalio autore.

\item Da li je struktura rada adekvatna?\\

Struktura rada je i više nego adekvatna. Svaka od stavki navedenih u sadržaju je, čini se, sa razlogom tu i služi svojoj svrsi. Ne vidim da bi išta što je objašnjeno u radu moglo biti suvišno.

\item Da li rad sadrži sve elemente propisane uslovom seminarskog rada (slike, tabele, broj strana...)?\\

Rad zadovoljava svaki od zahteva u pogledu broja slika, tabela, grafikona i strana koje treba da sadrži. 
Takođe, čak i da rad slučajno sadrži veći broj od broja dozvoljenih strana, zbog značaja obrađene teme 
koji sam već pomenuo, kao recenzent im to ne bih zamerio. 

Napomena: \\
Jedini uslov koji rad ne zadovolja jeste taj da u literaturi nije navedena nijedna knjiga, niti naučni 
članak, ali činjenica je da za obradu ove teme ništa od navedenog nije bilo ni potrebno. Takođe, 
navedeno je dosta konkursa čime je to nekako nadomešćeno.

\odgovor{Dodate su reference na knjigu i naučni članak.}

\item Da li su slike i tabele funkcionalne i adekvatne?\\

Po mom mišljenju su i više nego funkcionalne. Svaka od slika, tabela, grafika odgovara interesantnim 
podacima, tačnije, podacima iz interesantnih izvora i svaka od njih je ilustrativna i doprinosi 
sadržajnosti i estetici seminarskog rada zbog čega bih još jednom pohvalio autore.

\end{enumerate}

\section{Ocenite sebe}
Smatram sebe nedovoljno upućenim u oblast. Međutim, smatram da za recenziranje ovog rada recenzent 
(samim tim ni čitalac) ne mora da bude ekspert u datoj oblasti niti veoma upućen u njuI da bi mogao da autorima 
rada iznese neki konstruktivan predlog ili kritiku. Čini se da su svi podaci u ovom radu lako proverivi.
% Napišite koliko ste upućeni u oblast koju recenzirate: 
% a) ekspert u datoj oblasti
% b) veoma upućeni u oblast
% c) srednje upućeni
% d) malo upućeni 
% e) skoro neupućeni
% f) potpuno neupućeni
% Obrazložite svoju odluku



\chapter{Dodatne izmene}
%Ovde navedite ukoliko ima izmena koje ste uradili a koje vam recenzenti nisu tražili. 

\end{document}
